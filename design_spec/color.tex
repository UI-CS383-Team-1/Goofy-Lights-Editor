%%%%%%%%%%%%%%%%%%%%%%%%%%%%%%%%%%%%%%%%%%%%%%
% Joshua Bonn
% March 8, 2017
% Design specs for color selection and editing
%%%%%%%%%%%%%%%%%%%%%%%%%%%%%%%%%%%%%%%%%%%%%%

\documentclass[12pt]{article}

\begin{document}
 
\section{Color Editor}
The user will have the ability to select any color from the color wheel. He can also select from an preset selection of common colors.
\subsection{Color Selection}
There will be two modes of color selection: the color wheel, and a list of color presets.
\subsubsection{Color Wheel}
The user will click their mouse on the color wheel to select a base color. Then, he can adjust the saturation and gray-scale by dragging his mouse within a colored box. He will see the currently selected color in a box nearby. 
\subsubsection{Color Presets} 
There will be a list of twelve preset colors. The user can click these squares to automatically assign a color.
\subsection{Assigning Color}
The user will be able to assign color to a square in the current state by left clicking the square with their mouse. He can clear a square of any color by right clicking on a square. If no color is assigned to a square, the default will be no color.
\section{Additional Features}
\subsection{Paint Brush Coloring}
The user will be able to hold left click and drag their mouse over squares. This will paint all the squares it touches. The user can also hold right click to erase the squares.
\subsection{Assigning Personal Presets}
The user will be able to assign personal colors to the presets. The user will select a color on the color wheel. Then, the user will right click on a preset to assign the current color to that preset.

\end{document}
