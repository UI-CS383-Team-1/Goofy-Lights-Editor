%%%%%%%%%%%%%%%%%%%%%%%%%%%%%%%%%%%%%%%%%%%%%%%%%%%%%%%%%%%%%%%%%%%%%%%%%%%%%%%%%%%%%%%%%%%%%%%%%%
%% This file contains a description of the stamps function.
%% Christian Baisch
%% CS 383
%% 9 March 2017
%%%%%%%%%%%%%%%%%%%%%%%%%%%%%%%%%%%%%%%%%%%%%%%%%%%%%%%%%%%%%%%%%%%%%%%%%%%%%%%%%%%%%%%%%%%%%%%%%%

\documentclass{article}

\begin{document}
	\section{Stamps}
	The stamps function will allow the user to access previously saved images in a quick manner. 

	
	\subsection{Functionality}
	When clicking on the "stamps" icon, a box will pop up with different previously saved stamps. The user       will then click on one and it will be added to the current frame. Preset stamps include:
      \begin{itemize}
        \item Shapes
        \item Letters
        \item Numbers
      \end{itemize}
   \subsection{Creating Stamps}
   The user will also be able to add stamps to the program. A "save as stamp" option will be added to the      tool bar. This allows for the current frame to be added to the stamp feature.
	
\end{document}